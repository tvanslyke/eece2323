\documentclass[12pt, letterpaper]{article}

\usepackage{todonotes}
\usepackage{graphicx}
\usepackage{titling}
\usepackage{xcolor}
\usepackage{appendix}
\usepackage{setspace}
% so that we can use \FloatBarrier to keep figures from being
% moved to weird spots
\usepackage{placeins}
% Palatino font for text and math 
\usepackage{mathpazo}
% Bera Mono font for monospace text (code blocks)
\usepackage[scaled]{beramono}
\usepackage[T1]{fontenc}

% Syntax highlighting for Verilog code snippets
\usepackage{listings}
\definecolor{vgreen}{RGB}{104,180,104}
\definecolor{vblue}{RGB}{49,49,255}
\definecolor{vorange}{RGB}{255,143,102}
% style definition for verilog code
\lstdefinestyle{verilog-style}
{
    language=Verilog,
    basicstyle=\footnotesize\ttfamily,
    keywordstyle=\color{vblue},
    identifierstyle=\color{black},
    commentstyle=\color{vgreen},
    tabsize=2,  % small tab size so code doesn't run off the right-hand-side margin
    literate=*{:}{:}1
}


% Simple command to make figures for an image
\newcommand{\InsertImage}[3][\linewidth]{
	% Argument #1 (optional) = width of the image
	% Argument #2 = path to image
	% Argument #3 = caption for the image
	\begin{figure}[h]
		\centering
		\includegraphics[width=#1]{#2}
		\caption{#3}
	\end{figure}
}	

% Same as '\InsertImage', but with \FloatBarriers to keep the 
% figure from being moved around
\newcommand{\InsertImageHere}[3][\linewidth]{
	\FloatBarrier
	\InsertImage[#1]{#2}{#3}
	\FloatBarrier
}



\title{Lab \# 1-3: \textbf{8-Bit Adder, Partial Arithmetic Logic Unit, and Arithmetic Logic Unit}}

% I just copied your name as it appeared in gmail - Tim
\author{Group \# \textbf{\color{red}???}:\\ Jin Hyeong Kim \and\\ Timothy VanSlyke}


\begin{document}

% Title Page 
\begin{titlepage}
	\begin{center}
		{\Large
			\textbf{Northeastern University}\\
			~\\
			Department of Electrical and Computer Engineering\\ 
		}

		\vfill

		{\large
			EECE2323: \textbf{Digital Systems Design Lab}\\
			~\\
			Lecturer: \textbf{Dr. Emad Aboelela}\\
			~\\
			TAs:\\
			\textbf{Ke Chen}\\
			\textbf{Linbin Chen}\\
		}
	
		\vfill

		{\Large \thetitle}\\
	
		\vfill

		{\large \theauthor}\\

		\vfill

		{\large
			Semester: Spring 2018\\
			Date: \today\\
			Lab Session: Tuesday, 1:00PM\\ 
			Lab Location: 9 Hayden Hall, Northeastern University, Boston, MA 02115\\
		}

	\end{center}
\end{titlepage}

%% Table of Contents Page
\tableofcontents

%%%%%% INTRODUCTION %%%%%%
\newpage
\section{Introduction}
\begin{flushleft}
\doublespacing The objective of the series of Lab 1, 2, and 3 were to become familiarized with Verilog hardware descriptions (Xilinx) and use Verilog code to implmenet small features on Zedboard hardware. The goals of the following experiments were to implement a simple 8-bit adder, a partial arithmetic logic unit (ALU), and a full ALU by extending partial ALU. The Verilog code was first written and tested in the Xilinix platform, and then a bitstream was be generated for the hardware testing on the Zedboard, and functionality was verified on the board.
\end{flushleft}


%%%%%% DESIGN APPROACH %%%%%%
\newpage
\section{Design Approach}

\subsection{Software Used}
\begin{itemize}
	\item Xilinx Vivado - 
	\item Icarus Verilog - 
\end{itemize}

\subsection{Logic Design in Verilog Code}
The purpose of an 8-bit adder was to have two inputs of signed 8-bit and produce a signed 8-bit output and an overflow. Overflow was calculated using:
\begin{equation}
  ovf = (f[7] != a[7]) \quad\&\&\quad (f[7] != b[7])
\end{equation}
Appendix A shows details of the code used.

The partial ALU (PALU) had a 2-bit selector and two 8-bit inputs, and the 2-bit selector determined how the output is deteremind. If the selector is 00, the result would be the sum of two 8-bit inputs. If it is 01, the output will be a complement of input b. If it is 10, AND operator logic will apply to the inputs. Finally if it is 11, then OR operator logic will apply. Appendix B shows details of the code used.

Finally the full ALU had a 3-bit selector. Full ALU is an extension of partial ALU, as the selector now has 8 different states. In addition, the full ALU has one more output called $take\_branch$. The selector from 000 to 011 is same as PALU. If the selector is 100, the output is an arithmetic shift right of an input $a$. If it is 101, the output is a logic shift left of an input $a$. If it is 110, the output $f$ is 0, and the $take\_branch$ will be the result of $a == b$. If the selector is 111, the output $f$ will also be 0, but the $take\_branch$ will be the result of $a \neq b$. The full code is shown in Appendix C.

%%%%%% RESULTS AND ANALYSIS %%%%%%
\newpage
\section{Results and Analysis}




%%% DESIGN SIMULATION
\subsection{Design Simulation}


% lab 1
\subsubsection{8-Bit Adder}

\InsertImageHere{images/simulations/lab1/simulation.png}{Simulation results for lab 1, \textbf{8-Bit Adder}.}

% lab 2
\subsubsection{Partial Arithmetic and Logic Unit}

\InsertImageHere{images/simulations/lab2/simulation.png}{Simulation results for lab 2, \textbf{Partial Arithmetic and Logic Unit}.}

% lab 3
\subsubsection{Arithmetic and Logic Unit}

\InsertImageHere{images/simulations/lab3/simulation.png}{Simulation results for lab 3, \textbf{Arithmetic Logic Unit}.}


%%% HARDWARE TESTING
\subsection{Hardware Testing}


% lab 1
\subsubsection{8-Bit Adder}

% lab 2
\subsubsection{Partial Arithmetic and Logic Unit}

% lab 3
\subsubsection{Arithmetic and Logic Unit}




%%%%%% CONCLUSIONS %%%%%%
\newpage
\section{Conclusions}
The first three Lab sessions introduced the basics of Verilog and the functionalities of the hardware Zedboard. Implementation of logics with Verilog and verification of the logics through the FPGA on the ZedBoard were explored. The results came out successful. In future lab, the knowledge of thorough testing and effectively utilizing the software obtained from this lab would be useful.  




%%%%%% APPENDICES %%%%%%
\newpage
% start appendix 
\appendix
% explicit 'Appendix' in title before appendix sections
\appendixpage
% explicit 'Appendix' in table of contents
\addappheadtotoc 


\section{8-Bit Adder Module: \texttt{eightbit\_adder.v}}
\FloatBarrier
\begin{figure}[h]
	\lstinputlisting[style=verilog-style]{verilog/eightbit_adder.v}
	\caption{\texttt{eightbit\_adder.v} implementing an 8-bit adder in Verilog.}
\end{figure}
\FloatBarrier


\newpage
\section{Partual ALU Verilog Module: \texttt{eightbit\_palu.v}}
\FloatBarrier
\begin{figure}[h]
	\lstinputlisting[style=verilog-style]{verilog/eightbit_palu.v}
	\caption{\texttt{eightbit\_palu.v} implementing an 8-bit partial ALU in Verilog.}
\end{figure}
\FloatBarrier


\newpage
\section{ALU Verilog Module: \texttt{eightbit\_alu.v}}
\FloatBarrier
\begin{figure}[h]
	\lstinputlisting[style=verilog-style]{verilog/eightbit_alu.v}
	\caption{\texttt{eightbit\_alu.v} implementing an 8-bit ALU in Verilog.}
\end{figure}
\FloatBarrier


\newpage
\section{Hardware Test Results for Lab 1: 8-Bit Adder}
\FloatBarrier
\InsertImage[0.6\linewidth]{images/hardware/lab1/case-0.jpg}{First test case.}
\InsertImage[0.6\linewidth]{images/hardware/lab1/case-1.jpg}{Second test case.}
\FloatBarrier


\newpage
\section{Hardware Test Results for Lab 2: Partial Arithmetic Logic Unit}
\FloatBarrier
\InsertImage[0.6\linewidth]{images/hardware/lab2/case-0.jpg}{First test case.}
\InsertImage[0.6\linewidth]{images/hardware/lab2/case-1.jpg}{Second test case.}
\InsertImage[0.6\linewidth]{images/hardware/lab2/case-2.jpg}{Third test case.}
\InsertImage[0.6\linewidth]{images/hardware/lab2/case-3.jpg}{Fourth test case.}
\InsertImage[0.6\linewidth]{images/hardware/lab2/case-4.jpg}{Fifth test case.}
\FloatBarrier


\newpage
\section{Hardware Test Results for Lab 3: Arithmetic Logic Unit}
\FloatBarrier
\InsertImage[0.9\linewidth]{images/hardware/lab3/case-0.png}{First test case   -- $a + b$.}
\InsertImage[0.9\linewidth]{images/hardware/lab3/case-1.png}{Second test case  -- $\textasciitilde b$.}
\InsertImage[0.9\linewidth]{images/hardware/lab3/case-2.png}{Third test case   -- $a \cdot b$.}
\InsertImage[0.9\linewidth]{images/hardware/lab3/case-3.png}{Fourth test case  -- $a | b$.}
\InsertImage[0.9\linewidth]{images/hardware/lab3/case-4.png}{Fifth test case   -- $a >>> 1$.}
\InsertImage[0.9\linewidth]{images/hardware/lab3/case-5.png}{Sixth test case   -- $a << 1$.}
\InsertImage[0.9\linewidth]{images/hardware/lab3/case-6.png}{Seventh test case -- $beq(a, b)~~~(a \neq b)$.}
\InsertImage[0.9\linewidth]{images/hardware/lab3/case-7.png}{Eighth test case  -- $beq(a, b)~~~(a = b)$.}
\InsertImage[0.9\linewidth]{images/hardware/lab3/case-8.png}{Ninth test case   -- $bne(a, b)~~~(a = b)$.}
\FloatBarrier


\end{document}
