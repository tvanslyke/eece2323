\documentclass[12pt, letterpaper]{article}

\usepackage{todonotes}
\usepackage{graphicx}
\usepackage{titling}
\usepackage{xcolor}
\usepackage{appendix}

% for referencing items in the appendices with clickable links
\usepackage[ 
  colorlinks=true, 
  filecolor=cyan, 
]{hyperref}
% so that we can use \FloatBarrier to keep figures from being
% moved to weird spots
\usepackage{placeins}

% Palatino font for text and math 
\usepackage{mathpazo}

% Bera Mono font for monospace text (code blocks)
\usepackage[scaled]{beramono}
\usepackage[T1]{fontenc}

% Syntax highlighting for Verilog code snippets
\usepackage{listings}
\definecolor{vgreen}{RGB}{104,180,104}
\definecolor{vblue}{RGB}{49,49,255}
\definecolor{vorange}{RGB}{255,143,102}
% style definition for verilog code
\lstdefinestyle{verilog-style}
{
    language=Verilog,
    basicstyle=\footnotesize\ttfamily,
    keywordstyle=\color{vblue},
    identifierstyle=\color{black},
    commentstyle=\color{vgreen},
    tabsize=2,  % small tab size so code doesn't run off the right-hand-side margin
    literate=*{:}{:}1
}


% Simple command to make figures for an image
\newcommand{\InsertImage}[3][\linewidth]{
	% Argument #1 (optional) = width of the image
	% Argument #2 = path to image
	% Argument #3 = caption for the image
	\begin{figure}[h]
		\centering
		\includegraphics[width=#1]{#2}
		\caption{#3}
	\end{figure}
}	

% Same as '\InsertImage', but with \FloatBarriers to keep the 
% figure from being moved around
\newcommand{\InsertImageHere}[3][\linewidth]{
	\FloatBarrier
	\InsertImage[#1]{#2}{#3}
	\FloatBarrier
}



\title{Lab \# 1-3: \textbf{8-Bit Adder, Partial Arithmetic Logic Unit, and Arithmetic Logic Unit}}

% I just copied your name as it appeared in gmail - Tim
\author{Group \# \textbf{\color{red}???}:\\ Jin Hyeong Kim \and\\ Timothy VanSlyke}


\begin{document}

% Title Page 
\begin{titlepage}
	\begin{center}
		{\Large
			\textbf{Northeastern University}\\
			~\\
			Department of Electrical and Computer Engineering\\ 
		}

		\vfill

		{\large
			EECE2323: \textbf{Digital Systems Design Lab}\\
			~\\
			Lecturer: \textbf{Dr. Emad Aboelela}\\
			~\\
			TAs:\\
			\textbf{Ke Chen}\\
			\textbf{Linbin Chen}\\
		}
	
		\vfill

		{\Large \thetitle}\\
	
		\vfill

		{\large \theauthor}\\

		\vfill

		{\large
			Semester: Spring 2018\\
			Date: \today\\
			Lab Session: Tuesday, 1:00PM\\ 
			Lab Location: 9 Hayden Hall, Northeastern University, Boston, MA 02115\\
		}

	\end{center}
\end{titlepage}

%% Table of Contents Page
\tableofcontents

%%%%%% INTRODUCTION %%%%%%
\newpage
\section{Introduction}
The goal of the following experiments is implement a simple 8-bit Arithmetic Logic Unit (ALU) using the Verilog Hardware Description Language (HDL).  This implementation will be tested both in a software simulation use Xilinx Vivado, and in hardware on a ZedBoard with an integrated Field-Programmable Gate Array (FPGA)\footnote{The particular FPGA used is the Xilinx XC7Z020CLG484-1.}.  In our experiments, we bootstrap our design up from a simple adder circuit to an ALU capable of working with 8-bit integers, providing an instruction set of size 8.  

By exploring this process, we wish to explore the basics of digital design by providing a minimal example of an ALU.  \todo{Expand on this.}

%%%%%% DESIGN APPROACH %%%%%%
\newpage
\section{Design Approach}

\subsection{Software Used}
\begin{itemize}
	\item Xilinx Vivado - Xilinx Vivado was used to perform all model simulations, including testbenches.  Additionally, Xilinx Vivado provided the means with which our models were uploaded onto the ZedBoard FPGAs.
	\item Icarus Verilog - Icarus Verilog was used for rapid development and verification of several of the modules used in our experiments.
\end{itemize}

\subsection{Logic Design in Verilog Code}
Since both the partial \href{eightbit_partial_alu_module}{partial ALU} and \href{eightbit_alu_module}{full ALU} have an overflow (\texttt{ovf}) output bit that depends on the computed result of the ALU's \texttt{ADD} (\texttt{s}=0) operation, blocking assignment is used to compute \texttt{f} and \texttt{ovf}.  A non-blocking assignment would allow incorrect behavior by not obeying the dependency relationship between \texttt{f} and \texttt{ovf}.  It is worth noting that the \texttt{take\_branch} output bit in the full ALU could be set via a continuous assignment like so:
\begin{lstlisting}[style=verilog-style]
	// requires 'take_branch' to be declared as a 'wire'
	assign take_branch = ((s == 6) && (a == b)) || ((s == 7) && (a != b));
\end{lstlisting}
However, for the sake of consistency, we opt for a blocking assignment in the same \texttt{always} block as \texttt{f} and \texttt{ovf}.

The Verilog module definitions for the eight-bit adder, eight-bit partial ALU, and the eightbit ALU can respectively be found in the appendix sections \href{eightbit_adder_module}{A}, \href{eightbit_partial_alu_module}{B}, and \href{eightbit_alu_module}{C}.

Opcode dispatch is implemented in Verilog using a \texttt{case} statement which handles each instruction individually.  The \texttt{take\_branch} and \texttt{ovf} output bits in the eight-bit ALU are handled outside of the \texttt{case} statement, since they have consistent behavior in most cases.  For example, the \texttt{ovf} bit is set using the following statement:
\begin{lstlisting}[style=verilog-style]
	ovf = (s == 0) && (f[7] != a[7]) && (f[7] != b[7]);
\end{lstlisting}
This simplifies the \texttt{case} structure immensely; only \texttt{f} is mutated within it. 



%%%%%% RESULTS AND ANALYSIS %%%%%%
\newpage
\section{Results and Analysis}



%%% DESIGN SIMULATION
\subsection{Design Simulation}

% lab 1
\subsubsection{8-Bit Adder}

\InsertImageHere{images/simulations/lab1/simulation.png}{Simulation results for lab 1, \textbf{8-Bit Adder}.}

% lab 2
\subsubsection{Partial Arithmetic and Logic Unit}

\InsertImageHere{images/simulations/lab2/simulation.png}{Simulation results for lab 2, \textbf{Partial Arithmetic and Logic Unit}.}

% lab 3
\subsubsection{Arithmetic and Logic Unit}

\InsertImageHere{images/simulations/lab3/simulation.png}{Simulation results for lab 3, \textbf{Arithmetic Logic Unit}.}


%%% HARDWARE TESTING
\subsection{Hardware Testing}


% lab 1
\subsubsection{8-Bit Adder}

% lab 2
\subsubsection{Partial Arithmetic and Logic Unit}

% lab 3
\subsubsection{Arithmetic and Logic Unit}




%%%%%% CONCLUSIONS %%%%%%
\newpage
\section{Conclusions}




%%%%%% APPENDICES %%%%%%
\newpage
% start appendix 
\appendix
% explicit 'Appendix' in title before appendix sections
\appendixpage
% explicit 'Appendix' in table of contents
\addappheadtotoc 


\section{8-Bit Adder Module: \texttt{eightbit\_adder.v}} \label{eightbit_adder_module}
\FloatBarrier
\begin{figure}[h]
	\lstinputlisting[style=verilog-style]{verilog/eightbit_adder.v}
	\caption{\texttt{eightbit\_adder.v} implementing an 8-bit adder in Verilog.}
\end{figure}
\FloatBarrier


\newpage
\section{Partual ALU Verilog Module: \texttt{eightbit\_palu.v}} \label{eightbit_partial_alu_module}
\FloatBarrier
\begin{figure}[h]
	\lstinputlisting[style=verilog-style]{verilog/eightbit_palu.v}
	\caption{\texttt{eightbit\_palu.v} implementing an 8-bit partial ALU in Verilog.}
\end{figure}
\FloatBarrier


\newpage
\section{ALU Verilog Module: \texttt{eightbit\_alu.v}} \label{eightbit_alu_module}
\FloatBarrier
\begin{figure}[h]
	\lstinputlisting[style=verilog-style]{verilog/eightbit_alu.v}
	\caption{\texttt{eightbit\_alu.v} implementing an 8-bit ALU in Verilog.}
\end{figure}
\FloatBarrier


\newpage
\section{Hardware Test Results for Lab 1: 8-Bit Adder}
\FloatBarrier
\InsertImage[0.6\linewidth]{images/hardware/lab1/case-0.jpg}{First test case.}
\InsertImage[0.6\linewidth]{images/hardware/lab1/case-1.jpg}{Second test case.}
\FloatBarrier


\newpage
\section{Hardware Test Results for Lab 2: Partial Arithmetic Logic Unit}
\FloatBarrier
\InsertImage[0.6\linewidth]{images/hardware/lab2/case-0.jpg}{First test case.}
\InsertImage[0.6\linewidth]{images/hardware/lab2/case-1.jpg}{Second test case.}
\InsertImage[0.6\linewidth]{images/hardware/lab2/case-2.jpg}{Third test case.}
\InsertImage[0.6\linewidth]{images/hardware/lab2/case-3.jpg}{Fourth test case.}
\InsertImage[0.6\linewidth]{images/hardware/lab2/case-4.jpg}{Fifth test case.}
\FloatBarrier


\newpage
\section{Hardware Test Results for Lab 3: Arithmetic Logic Unit}
\FloatBarrier
\InsertImage[0.9\linewidth]{images/hardware/lab3/case-0.png}{First test case   -- $a + b$.}
\InsertImage[0.9\linewidth]{images/hardware/lab3/case-1.png}{Second test case  -- $\textasciitilde b$.}
\InsertImage[0.9\linewidth]{images/hardware/lab3/case-2.png}{Third test case   -- $a \cdot b$.}
\InsertImage[0.9\linewidth]{images/hardware/lab3/case-3.png}{Fourth test case  -- $a | b$.}
\InsertImage[0.9\linewidth]{images/hardware/lab3/case-4.png}{Fifth test case   -- $a >>> 1$.}
\InsertImage[0.9\linewidth]{images/hardware/lab3/case-5.png}{Sixth test case   -- $a << 1$.}
\InsertImage[0.9\linewidth]{images/hardware/lab3/case-6.png}{Seventh test case -- $beq(a, b)~~~(a \neq b)$.}
\InsertImage[0.9\linewidth]{images/hardware/lab3/case-7.png}{Eighth test case  -- $beq(a, b)~~~(a = b)$.}
\InsertImage[0.9\linewidth]{images/hardware/lab3/case-8.png}{Ninth test case   -- $bne(a, b)~~~(a = b)$.}
\FloatBarrier


\end{document}
