\documentclass{article}
\title{Prelab 4}
\author{Timothy Van Slyke}
\usepackage{placeins}
\usepackage{xcolor}
\usepackage{listings}
\definecolor{vgreen}{RGB}{104,180,104}
\definecolor{vblue}{RGB}{49,49,255}
\definecolor{vorange}{RGB}{255,143,102}

\lstdefinestyle{verilog-style}
{
    language=Verilog,
    basicstyle=\small\ttfamily,
    keywordstyle=\color{vblue},
    identifierstyle=\color{black},
    commentstyle=\color{vgreen},
    tabsize=2,  % small tab size so code doesn't run off the right-hand-side margin
    literate=*{:}{:}1
}

\begin{document}
\maketitle
\section{Part 1}
\begin{lstlisting}[style=verilog-style]
module reg_file(
	input rst, 
	input clk, 
	input wr_en, 
	input [1:0] rd0_addr, 
	input [1:0] rd1_addr, 
	input [1:0] wr_addr, 
	input [8:0] wr_data,
	output reg [8:0] rd0_data,
	output reg [8:0] rd1_data
);
	reg [3:0] [8:0] storage;
	
	always @(posedge clk) begin
		if(rst) begin
			storage[0] <= 0;
			storage[1] <= 0;
			storage[2] <= 0;
			storage[3] <= 0;
		end else if(wr_en) begin
			storage[wr_addr] <= wr_data;
		end else begin
			rd0_data <= storage[rd0_addr];
			rd1_data <= storage[rd1_addr];
		end
	end
endmodule
\end{lstlisting}


\section{Test Cases}
\begin{figure}[h]
	\begin{tabular}{r r r r r r}
		time (ns) & ALU operand 1 & ALU operand 2 & ALU output & ALU ovf & ALU take\_branch\\
		\hline
		  0 & xxxxxxxx      & xxxxxxxx      & xxxxxxxx   & x       & x\\  
		 10 & xxxxxxxx      & xxxxxxxx      & xxxxxxxx   & x       & 0\\  
		 20 & xxxxxxxx      & xxxxxxxx      & xxxxxxxx   & x       & 0\\  
		 30 & xxxxxxxx      & xxxxxxxx      & xxxxxxxx   & x       & 0\\  
		 40 & 10000000      & 10000001      & 00000001   & 1       & 0\\  
		 50 & 10000000      & 10000001      & 00000001   & 1       & 0\\  
		 60 & 10000000      & 11100111      & 01100111   & 1       & 0\\  
		 70 & 10000000      & 11100111      & 01100111   & 1       & 0\\  
		 80 & 10000000      & 11100111      & 01100111   & 1       & 0\\  
		 90 & 10000000      & 11100111      & 01100111   & 1       & 0\\  
		100 & 10000000      & 11100111      & 01100111   & 1       & 0\\  
		110 & 00000000      & 11100111      & 11100111   & 0       & 0\\  
		120 & 10000000      & 11111111      & 01111111   & 1       & 0\\  
		130 & 00000000      & 11111111      & 11111111   & 0       & 0\\  
		140 & 10000000      & 11100111      & 01100111   & 1       & 0\\ 
	\end{tabular}
	\caption{Outputs for each test vector.}
\end{figure}

\begin{figure}[h]
	\begin{tabular}{r r r r r r r r}
		time (ns) & rst & write enable & write data & instr\_i  & mux 1 select & mux 2 select & ALU op\\
		\hline
		  5 & 0   & 0            & 000000000  & 00000000 & 0            & 0            & 000\\ 
		 15 & 0   & 1            & 010000000  & 00000000 & 0            & 0            & 000\\ 
		 25 & 0   & 1            & 010000001  & 00000000 & 0            & 0            & 000\\ 
		 35 & 0   & 0            & 010000001  & 00000000 & 0            & 0            & 000\\ 
		 45 & 0   & 1            & 011100111  & 00000000 & 0            & 0            & 000\\ 
		 55 & 0   & 0            & 011100111  & 00000000 & 0            & 0            & 000\\ 
		 65 & 1   & 0            & 011100111  & 00000000 & 0            & 0            & 000\\ 
		 75 & 0   & 1            & 011100111  & 00000000 & 0            & 0            & 000\\ 
		 85 & 0   & 1            & 010000000  & 00000000 & 0            & 0            & 000\\ 
		 95 & 0   & 0            & 010000000  & 11111111 & 0            & 0            & 000\\ 
		105 & 0   & 0            & 010000000  & 11111111 & 1            & 0            & 000\\ 
		115 & 0   & 0            & 010000000  & 11111111 & 0            & 1            & 000\\ 
		125 & 0   & 0            & 010000000  & 11111111 & 1            & 1            & 000\\ 
		135 & 0   & 0            & 010000000  & 11111111 & 0            & 0            & 000\\ 
		145 & 0   & 0            & 010000000  & 11111111 & 0            & 0            & 000\\ 
		\end{tabular}
		\caption{Inputs for each test vector.}
\end{figure}


\FloatBarrier

\section*{Other Modules Used for Simulation}
\begin{lstlisting}[style=verilog-style]
module alu_reg(
	// inputs
	input rst, 
	input clk, 
	input RegWrite, 
	input [1:0] ReadAddr1, 
	input [1:0] ReadAddr2, 
	input [1:0] WriteAddr, 
	input [8:0] WriteData,
	input [7:0] Instr_i,
	input ALUSrc1,
	input ALUSrc2,
	input [2:0] ALUOp,
	// outputs
	output reg [7:0] result,
	output wire [7:0] input1,
	output reg [7:0] input2,
	output reg ovf,
	output reg take_branch
);
	// reg file outpus
	reg [8:0] reg_data0;
	reg [8:0] reg_data1;
	// mux'd inputs to alu

	// register file instantiation
	reg_file RegFile(
		.rst(rst),
		.wr_en(RegWrite),
		.clk(clk),
		.rd0_addr(ReadAddr1),
		.rd1_addr(ReadAddr2),
		.wr_addr(WriteAddr),
		.wr_data(WriteData),
		.rd0_data(reg_data0),
		.rd1_data(reg_data1)
	);
	
	assign input1 = ALUSrc1 ? 8'b0000_0000 : reg_data0; 
	assign input2 = ALUSrc2 ? Instr_i : reg_data1;

	// ALU instantiation
	eightbit_alu EightbitALU(
		.a(input1),
		.b(input2),
		.s(ALUOp),
		.f(result),
		.ovf(ovf),
		.take_branch(take_branch)
	);

endmodule
\end{lstlisting}

\begin{lstlisting}[style=verilog-style]
module alu_reg_tb();
	reg clk;
	always #5 clk = !clk;
	reg rst;
	reg write_enable;
	reg [1:0] read_address1;
	reg [1:0] read_address2;
	reg [1:0] write_address;
	reg [8:0] write_data;
	reg [7:0] instr_i;
	reg alu_src1;
	reg alu_src2;
	reg [2:0] alu_op;

	wire [7:0] alu_result;
	wire [7:0] alu_input1;
	wire [7:0] alu_input2;
	wire ofv;
	wire take_branch;

	alu_reg AluReg(
		// inputs
		.rst(rst), 
		.clk(clk), 
		.RegWrite(write_enable), 
		.ReadAddr1(read_address1), 
		.ReadAddr2(read_address2), 
		.WriteAddr(write_address), 
		.WriteData(write_data),
		.Instr_i(instr_i),
		.ALUSrc1(alu_src1),
		.ALUSrc2(alu_src2),
		.ALUOp(alu_op),
		// outputs
		.result(alu_result),
		.input1(alu_input1),
		.input2(alu_input2),
		.ovf(ovf),
		.take_branch(take_branch)
	); // AluReg();

	integer test_duration = 150;
	
	always @(posedge clk) begin
		if ($time < test_duration) begin
			$display("Clock Tick (posedge t = %d): ", $time);
			$display({"Inputs:\n",
				"rst = %b, ",
				"write enable = %b, ",
				"write data = %b, ",
				"instr_i = %b, ",
				"mux 1 select = %b, ",
				"mux 2 select = %b, ",
				"ALU op = %b, "
			}, rst, write_enable, write_data, instr_i, alu_src1, alu_src2, alu_op);
		end
	end
	always @(negedge clk) begin
		if ($time < test_duration) begin
			$display("Clock Tick (negedge t = %d): ", $time);
			$display({"Ouputs:\n",
				"ALU operand 1 = %b, ",
				"ALU operand 2 = %b, ",
				"ALU output = %b, ",
				"ALU ovf = %b, ",
				"ALU take_branch = %b, "
			}, alu_input1, alu_input2, alu_result, ovf, take_branch);
		end
	end
	initial begin
		clk = 0;
		rst = 0;
		read_address1 = 0;
		read_address2 = 1;
		write_enable = 1;
		write_address = 0;
		write_data = 0;
		instr_i = 0;
		alu_src1 = 0;
		alu_src2 = 0;
		alu_op = 0;
		#10  // load value into addr 0
		write_address = 0;
		write_data = 8'b1000_0000;
		#10 // load value into addr 1
		write_address = 1;
		write_data = 8'b1000_0001;
		#10 // read values into respective operands
		write_enable = 0;
		read_address1 = 0;
		read_address2 = 1;
		#10 // load into address 2 
		write_enable = 1;
		write_address = 2;
		write_data = 8'b1110_0111;
		#10 // read address 2 into operand 2
		write_enable = 0;
		read_address2 = 2;
		#10 // test reset functionality
		rst = 1;
		#10 //load values back into registers
		rst = 0;
		write_enable = 1;
		write_address = 2;
		write_data = 8'b1110_0111;
		#10  
		write_address = 0;
		write_data = 8'b1000_0000;
		#10 // read values we just loaded into operands
		write_enable = 0;
		read_address1 = 0;
		read_address2 = 2;
		
		// test multiplexers
		instr_i = 8'b1111_1111;
		#10
		alu_src1 = 1;
		alu_src2 = 0;
		#10
		alu_src1 = 0;
		alu_src2 = 1;
		#10
		alu_src1 = 1;
		alu_src2 = 1;
		#10
		alu_src1 = 0;
		alu_src2 = 0;
		#10
		alu_op = 0; // no-op
		// test ALU opcodes
		// #10
		// alu_op = 1;
		// #10
		// alu_op = 1;
		// #10
		// alu_op = 2;
		// #10
		// alu_op = 3;
		// #10
		// alu_op = 4;
		// #10
		// alu_op = 5;
		// #10
		// alu_op = 6;
		// #10
		// alu_op = 7;
		// #10
		// alu_op = 7; // noop
	end

endmodule
\end{lstlisting}

\end{document}
